\documentclass{article}
\usepackage[utf8]{inputenc} 
\usepackage{ulem}
\usepackage{graphicx}
\usepackage{parskip}
\usepackage{enumitem}
\usepackage[russian]{babel}
\usepackage[a4paper, left=2cm, right=2cm, top=2cm, bottom=2cm]{geometry} 
\usepackage{multicol} 
\begin{document}
\begin{multicols}{2} 
\noindent
\normalsize
and others are violated. These include relevance logics
and connexive logics which find out to justify causal
implicative properties. Analyzing the properties of these
logics involves clarifying the similarities and analogies 
of the schemes of these logics with other logics and
models such as argumentation logics [2]. One of the
prospects for further research is to study the connection
of non-classical logics of this kind with the fuzzy models
considered in this paper in the framework of causal and
spatio-temporal relations of the semantic space.

\begin{center}
\large
III. Conclusions \par
\end{center}
\normalsize

Approaches and models to the interpretation of fuzzy
logics are proposed. The proposed models can be used
in the interpretation of fuzzy logic formulas on the basis
of metric meaning space for finite structures in order to
analyze or synthesize schemes of fuzzy logic inference
systems relevant to the structures of ontologies of subject
areas.

\begin{center}
\large
References
\end{center}
\begin{enumerate}[label={[\arabic*]},itemsep=0pt, leftmargin=2em, labelsep=1em,topsep=0pt]
\footnotesize
\setlength{\parskip}{0pt}
\item J.M. Lee. Introduction to Topological Manifolds, Springer, 2011,
452 p.
\noindent\item Finn V. K. Intellektual’nye sistemy i obshchestvo: Sbornik statei
[Intelligent systems and society: Collection of articles], Moscow,
KomKniga, 2006. 352 p.
\noindent\item D.E. Pal’chunov, G.E. Yakh"yaeva Nechetkie logiki i teoriya
nechetkikh modelei [Fuzzy logic and theory of fuzzy models],
Algebra i logika [Algebra and Logic], vol. 54 no.1, 2015, p. 109–
118; Algebra and Logic, vol. 54, no. 1, 2015, p. 74–80.
\noindent\item Plesnevich, G.S. Binarnye modeli znanii [Binary knowledge
models], Trudy Mezhdunarodnykh nauchno-tekhnicheskikh konferentsii «Intellektual’nye sistemy» (AIS’08) i «Intellektual’nye
SAPR» (CAD-2008) [Proceedings of the International Scientific and Technical Conferences “Intelligent Systems” (AIS’08)
and "Intelligent CAD" (CAD-2008)]. Nauchnoe izdanie v 4-kh
tomakh [4 volumes], Moscow, Fizmatlit, 2008, vol.2. p. 424, 135–
146 pp.
\noindent \item V.N. Vagin Deduktsiya i obobshchenie v sistemakh prinyatiya
reshenii [Deduction and generalization in decision-making systems], Moscow, Nauka. Gl. red. fiz.-mat. lit. 1988. 384 p.
\noindent\item  V.P. Ivashenko Ontologicheskie struktury i parametrizovannye
mnogoznachnye logiki [Ontological structures and parameterized
multivalued logics], Information Technologies and Systems, 2023,
Minsk, BGUIR, pp. 57–58.
\noindent\item V. Ivashenko Semantic space integration of logical knowledge
representation and knowledge processing models Otkrytye semanticheskie tekhnologii proektirovaniya intellektual’nykh system [Open semantic technologies for intelligent systems], Minsk,
BGUIR. Minsk, 2023, vol. 7. pp. 95–114.
\noindent\item Ivashenko, V. General-purpose semantic representation language
and semantic space Otkrytye semanticheskie tekhnologii proektirovaniya intellektual’nykh system [Open semantic technologies
for intelligent systems] Minsk, BGUIR, 2022. vol. 6. pp. 41–64.
\noindent\item G. Malinowski, Kleene logic and inference. Bulletin of the Section
of Logic, 2014, vol. 1, no. 43.
\noindent\item D. Maximov, Logika N.A. Vasil’eva i mnogoznachnye logiki
[Vasil’ev Logic and Multi-Valued Logics.] Logical Investigations,
2016, vol. 22. 82–107 pp.
\noindent \item V.P. Ivashenko. Modeli resheniya zadach v intellektual’nykh sistemakh. V 2 ch. Ch. 1 : Formal’nye modeli obrabotki informatsii
i parallel’nye modeli resheniya zadach : ucheb.-metod. posobie
[Models for solving problems in intelligent systems. In 2 parts,
Part 1: Formal models of information processing and parallel
models for solving problems: a tutorial] Minsk, BGUIR, 2020.
79 p.

\noindent \item A. Hussain, K. Ullah, M. Khan, T. Senapati, S. Moslem,
Complex T-Spherical Fuzzy Frank Aggregation Operators With
Application in the Assessment of Soil Fertility / IEEE Access,
2023, vol. 11.
\noindent \item E.P. Klement, M. Navara, Propositional Fuzzy Logics Based
on Frank T-Norms: A Comparison // Fuzzy Sets, Logics and
Reasoning about Knowledge 2000, vol. 15. 17–38 pp.
\noindent \item F. Giannini, M. Diligenti, M. Maggini, M. Gori, G. Marra,
T-norms driven loss functions for machine learning. Applied
Intelligence, 2023, vol. 53. 1–15 pp.
\noindent \item G. Heald Issues with reliability of fuzzy logic. Int. J. Trend Sci.
Res. Develop, 2018, vol. 2 no. 6, 829-834 pp. 8, 2018.
\noindent \item Zh. Wang, J. K. George Fuzzy Measure Theory, Plenum Press,
New York, 1991.
\noindent \item V.V. Golenkov. Otkrytyi proekt, napravlennyi na sozdanie
tekhnologii komponentnogo proektirovaniya intellektual’nykh sistem [An open project aimed at creating a technology for the
component design of intelligent systems], Otkrytye semanticheskie tekhnologii proektirovaniya intellektual’nykh system [Open
semantic technologies for intelligent systems], 2013, pp. 55—78.
\noindent \item A.S. Narinyani. NE-faktory: netochnost’ i nedoopredelennost’ –
razlichie i vzaimosvyaz’ [Non-factors: inaccuracy and underdetermination – difference and interrelation]. Izv RAN (RAS). Ser.
Teoriya i sistemy upravleniya 5, 2000. pp. 44—56.
\noindent \item Yu. Manin, M. Marcolli. Semantic spaces. Published, Location,
2016. 32 p. (arXiv)
\noindent \item D.A. Pospelov. Situatsionnoe upravlenie: teoriya i praktika [Situational management: theory and practice], Moscow, Nauka, 1986.
288 p.
\end{enumerate}
\begin{center}
\noindent
\large
\textbf{ИНТЕГРАЦИЯ НЕЧЁТКИХ СИСТЕМ И
ИХ ПАРАМЕТРИЧЕСКАЯ
ИНТЕРПРЕТАЦИЯ ДЛЯ
УНИФИЦИРОВАННОГО
ПРЕДСТАВЛЕНИЯ ЗНАНИЙ}
\end{center}
\begin{center}\large\textbf{
  Ивашенко В. П.  }
\end{center}
\normalsize
В статье рассматривается проблема устойчивой интерпретации нечётких логических моделей. Предлагается подход на основе параметризованной нечёткой
логики, где каждая логическая формула кроме значений истинности имеет набор модельных параметров. Параметризованные нечёткая логика позволяет
объединить различные нечёткие логические системы.
Модельные параметры используются для вычисления
значений нечёткой истинности, как нечёткой меры.
Рассмотрены модели и модельные параметры, связанные с метрическими пространствами, согласуемыми с
метрическим смысловыми пространствами и являющиеся основой для интерпретации нечётких логических формул на онтологических моделях.\par
\begin{flushright}
Received 03.04.2024
\end{flushright}
\newpage
\end{multicols}

\begin{center}
\Huge \textbf{Current State of ostis-systems Component
Design Automation Tools} 
\end{center}
\begin{center}
    Maksim Orlov, Anna Makarenko, Ksenija Petrochuk\par
\textit{Belarusian State University of} \par
\textit{Informatics and Radioelectronics} \par
 Minsk, Belarus\par
Email: orlovmaksimkonst@gmail.com, anna.makarenko1517@gmail.com, xenija.petrotschuk@gmail.com
\end{center}
\begin{multicols}{2}
\normalsize
\setlength{\parindent}{1em}
 \textbf {\textit{Abstract}—In the article, an approach to the design of
intelligent systems is considered, focused on the use of compatible reusable components, which significantly reduces
the complexity of developing such systems. The key means
of supporting the component design of intelligent computer
systems is the manager of reusable components proposed
    in the work.\par}
    \textbf{
\textit{Keywords}—Component design of intelligent computer
systems; reusable semantically compatible components;
knowledge-driven systems; semantic networks; OSTIS Technology.}
\normalfont
\begin{center}
    I. Introduction\par
\end{center}

The main result of artificial intelligence is not the
intelligent systems themselves, but powerful and effective technologies for their development. The analysis of
modern technologies for designing intelligent computer
systems shows that along with very impressive achievements, the following serious problems occur [1]–[3]:

\begin{itemize}[noitemsep]
\setlength{\parskip}{0pt}
    \item \uline{high} requirements for the initial qualifications of
users and developers. Artificial intelligence technologies are not focused on the \uline{wide} range of
developers and users of intelligent systems and,
therefore, have not received mass distribution;
    \item modern information technologies are not oriented
to a wide range of developers of applied computer
systems;
    \item there is no general-unified solution to the problem of \uline{semantic compatibility} of computer systems
[4]. There are no approaches that allow integrating
scientific and practical results in the field of artificial intelligence, which generates a high degree
of \uline{duplication} of results and a lot of non-unified
formats for representation of data, models, methods,
tools, and platforms;
\item lack of powerful tools for designing intelligent computer systems, including intelligent training subsystems, subsystems for collective design of computer
systems and their components, subsystems for verification and analysis of computer systems, \uline{subsystems
for component design of computer systems;}
\item  \uline{long} terms of development of intelligent computer
systems and \uline{high} level of complexity of their maintenance and extension;
\item the degree of dependence of artificial intelligence
technologies on the \uline{platforms} on which they are
implemented is high, which leads to significant
changes in technologies when transitioning to new
platforms;
\item the degree of dependence of artificial intelligence
technologies on \uline{subject domains} in which these
technologies are used is high;
\item there is a high degree of dependence of intelligent
computer systems and their components on each
other; the lack of their \uline{automatic} synchronization.
The absence of self-sufficiency of systems and components, their ability to operate separately from each
other without loss of expediency of their use;
\item increase in the time to solve the problem with the
expansion of the functionality of the problem solver
and with the expansion of the knowledge base of
the system [5];
\item lack of methods for designing intelligent computer
systems. Updating computer systems often boils
down to the development of various kinds of
“patches”, which eliminate not \uline{causes} of the identified disadvantages of updated computer systems
but only some consequences of these causes;
\item poor adaptability of modern computers to the effective implementation of even existing knowledge
representation models and models for solving problems that are difficult to formalize, which requires
the development of \uline{fundamentally} new computers
[6];
\item there is no single approach to the allocation of
\uline{reusable components} and the formation of libraries
of such components, which leads to a high complexity of reuse and integration of previously developed
components in new computer systems.
\item there is a variety of semantically equivalent implementations of problem-solving models, duplication
of knowledge base and user interface components
that differ not in the essence of these components
but in the form of representation of the processed
information;


\end{itemize}
To solve these problems,it is necessary to implement a
\newpage

\columnbreak

comprehensive technology for designing intelligent computer systems, which includes the following components:
\begin{itemize}[noitemsep]
\item a model of an intelligent computer system [7];
\item \textit{a library of reusable components and corresponding
tools to support component design of intelligent
computer systems;}
\item  an intelligent integrated automation system for the
collective design of intelligent computer systems,
including subsystems for editing, debugging, performance evaluation, and visualization of developed
components, as well as a simulation subsystem;
\item methods of designing intelligent computer systems;
\item an intelligent user interface;
\item training subsystems for designing intelligent computer systems, including a subsystem for conducting
a dialogue with the developer and the user;
\item a subsystem for testing and verification of intelligent
computer systems, including a subsystem for testing
the compatibility of the developed system with other
systems;
\item an information security support subsystem for the
intelligent computer system.\par

\end{itemize}
The key component of the technology for intelligent
systems design is a component design that is represented
as a \textit{library of reusable components and the corresponding tools for supporting component design of intelligent
computer systems.} With its help, it is possible to effectively implement the typical subsystems to support the
design of intelligent computer systems.
\begin{center}
    II. Analysis of existing approaches to solving the\par
problem

\end{center}
\hspace*{2em}The problem is the lack of accessibility and integration
in artificial intelligence technologies, which have high
initial qualification requirements, lack a unified semantic
compatibility solution, and have a high degree of dependency on platforms, subject areas, and components,
leading to long development times, high maintenance
costs, and difficulties in reusing and integrating previously developed components in new systems.\par
Existing approaches to solving the problem include
component libraries and package managers of programming languages and operating systems, as well as separate systems and platforms with built-in components and
means for saving created components.\par
The components of the library may be implemented in
different programming languages (which leads to the fact
that for each programming language different libraries
are developed with their own solutions to various common situations), and may be located in different places,
which leads to the fact that the library needs a tool to
find components and install them.\par
Modern package managers such as \textit{npm, pip, papt,
maven, poetry} and others have the advantage that they
are able to resolve conflicts when installing dependent
components, but they do not take into account the
semantics of components, but only install components
by the [8] identifier. Libraries of such components are
only a repository of components, without taking into
account the purpose of components, their advantages
and disadvantages, areas of application, the hierarchy
of components and other information necessary for the
intellectualization of component design of computer systems. Searching for components in \textit{component libraries}
corresponding to these package managers is reduced
to searching by component identifier. Modern package
managers are only "installers" without automatic integration of components into the system. Also a significant
disadvantage of the modern approach is the platform
dependency of components. Modern component libraries
are oriented only to a certain programming language,
operating system or platform.\par
The \textbf{pip} package manager is a package management
system that is used to install packages from the Python
Package Index, which is some library of such packages.
Pip is often installed with Python. The pip package manager is used only for the Python programming language.
It has many functions for working with packages:
\begin{itemize}[noitemsep]
\item installation of a package;
\item installation of a package of a specialized version;
\item deletion of a package;
\item reinstallation of a package;
\item display of installed packages;
\item search for packages;
\item verification of package dependencies;
\item creation of a configuration file with a list of installed
packages and their versions;
\item installation of a set of packages from a configuration
file.

\end{itemize}
\includegraphics[scale=0.55]{image.png}
\begin{center}
    Figure 1. pip configuration file
\end{center}
\hspace*{2em}The pip package manager works well with dependencies, displays unsuccessfully installed packages, and also
displays information about the required package version
in case of conflict with another package. An example of
a pip package configuration file is shown in Figure 1.\par
Another example of a package manager is \textit{npm}. npm
is a package manager for the javaScript language. The
\end{multicols}
\end{document}